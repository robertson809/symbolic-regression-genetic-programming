\textsl{•}
%%%
% Any line that begins with a percent symbol is a comment. To compile
% this document and view the output:
%
% Run Latex
% Run Bibtex
% Then run Latex twice.
%
% This should produce the output PDF file named main.pdf
%%%

% This defines the style to use for this document.
% Do not modify.
\documentclass[letterpaper]{article}

% The following are akin to "import" statements in Python or Java -
% these import useful commands into the document for you to use.  You
% don't have to modify any of these lines. The AAAI package formats
% this document in the style of submissions to the American
% Association for Artificial Intelligence conference, one of the top
% AI conferences in the world. You will find that many academic
% publications in AI use this format.
\usepackage{aaai} 
\usepackage{times} 
\usepackage{helvet} 
\usepackage{courier} 
\setlength{\pdfpagewidth}{8.5in} 
\setlength{\pdfpageheight}{11in} 
\usepackage{amsmath}
\usepackage{amsthm}
\usepackage{graphicx}
\usepackage{graphics}
\usepackage{moreverb}
\usepackage{subfigure}
\usepackage{epsfig}
\usepackage{txfonts}
\usepackage{algpseudocode}
\usepackage{multirow, multicol}
\usepackage{url}
\usepackage{tablefootnote}
\usepackage{color}

\setcounter{secnumdepth}{1}
\nocopyright

% Fill in your paper title, names and emails below
% The "\\" is used to break lines. The \url command
% is useful for typesetting URLs and email addresses (it uses the
% Courier font).
\title{Solving the 8-Puzzle}
 \author{Alan Turing \and John von Neumann\\
 \url{{alturing, jovonneumann}@davidson.edu}\\
 Davidson College\\
 Davidson, NC 28035\\
 U.S.A.}

% This is the "true" start of the document. All the text in your
% write-up should be placed within the \begin{document} and
% \end{document} decorators.
\begin{document}

\maketitle % formats the title nicely, do not modify

% While at this point you could just begin your write-up, often, it's
% useful to write each section of your write-up in a separate tex
% file (not unlike the modular decomposition you do for code you
% write). These \input commands insert the contents of the
% specified tex files in the order specified. Every write-up you
% submit must contain the following sections, in the shown order. Open
% each of the indicated tex files to understand what goes in each
% section, as well as for more TeX tips.
\input{abstract}
\input{intro}
\input{background}

\section{Experiments}
\label{sec:expts}

The difficulties of genetic programming present themselves to the designers in the form of infamously many adjustable parameters, all of which influence the end efficacy of the program in interdependent and inscrutable ways. We attempted to find a choice of population size, method of original tree generation, tree size penalty, mutation rate, mutation method, parent selection method (with its own considerable set of variable parameters), and reproduction mechanism which both cause the program to converge to a solution and to maximize the rate of that convergence. 

We defined convergence as reaching a below [x] mean-squared error from the test data set, and failure to converge as moving through [x] generations without producing a new child more fit than any of its ancestors. We measured the success of our program by both the percentage of runs that converged, and the number of generations that convergence took. 


We investigated different methods of parent selection first, believe that parameter to influence the convergence most. Our research indicated that experts in the field used tournament selection and fitness-proportion ("roulette selection") most often. 

Our best results came from a [x] proportion of population size to tournament size, and when we allowed only [y] survivors for each tournament to reproduce randomly with themselves. 



Second, we ran fitness proportion selection, where we chose different methods for assigning reproduction probability.



For each of these methods, we allowed the selected parents to randomly reproduce with each other, producing one child per meeting, and repeated the selection process until the size of the child population reached the size of the original adult population. 


[Present results]
were the questions you were trying to answer? What was the
experimental setup (number of trials, parameter settings, etc.)? What
were you measuring? You should justify these choices when
necessary. The accepted wisdom is that there should be enough detail
in this section that I could reproduce your work \emph{exactly} if I
were so motivated.


\section{Results}
\label{sec:results}

We hope to have a success rate of a program with specific parameters defined as percentage of a total number of runs for which it reached below error tolerance, and we will further report the average number of generations needed for it to converge. Finally, we hope to report the equation which models the dataset with a mean-squared-error below our tolerance.  

  % replace the second argument below with your filename. I like to
  % place all my figures in a sub-directory to keep things organized
  \includegraphics[width=0.47\textwidth]{figs/results.pdf}

  % *Every* figure should have a descriptive caption.
  \caption{Figure 1}

  % The label is a handle you create so that you can refer to this
  % figure (using the \ref{} command) from other parts of your
  % document. LaTeX automatically renumbers figures and updates
  % references when you recompile, so you should do it this way rather
  % than hard-coding in references. Notice that I've also been
  % creating labels for the various sections in the document; I could
  % use \ref{} command to refer to those sections using their labels
  % too.
  \label{fig:tex}

\end{figure}

\subsection{Creating Tables}
\label{subsec:tables}

Again, refer to \texttt{results.tex} to learn how to create simple
tables (like table~\ref{tab:example}).
\begin{figure}[htb]
  \centering % centers the entire table

  % The following line sets the parameters of the table: we'll have
  % three columns (one per 'c'), each
  % column will be centered (hence the 'c'; 'l' or 'r' will left or
  % right justify the column) and the columns
  % will have lines between them (that's the purpose of the |s between
  % the 'c's).
  \begin{tabular}{|c|c|c|} 
    \hline \hline % draws two horizontal lines at the top of the table
    Column 1 & Column 2 & Column 3 \\ % separate column contents using the &
    \hline % line after the column headers
    $1$ & $3.1$ & $2.7$ \\
    $42$ & $-1$ & $1729$\\
    \hline \hline
  \end{tabular}

  % As with figures, *every* table should have a descriptive caption
  % and a label for ease of reference.
  \caption{An example table.}
  \label{tab:example}

\end{figure}


\input{conclusion}

\section{Contributions}
\label{sec:contrib}

Due to the large volume of coding in this project, the authors worked separately to develop moths of the classes and methods. Andrew wrote most of the functionality of the Tree, including the crossover, mutate, evaluation, and calculation of fitness methods. He also wrote functions to read the supplied data into a usable format.  Michael completed the basic implementation of the Tree that Andrew began, as well as the underlying Node class. Michael wrote the driver program to create new generations, move through generations, as well as the tournament selection method. Each partner researched the parts of the program the implemented.

\input{ack}

% This creates the references section. Open the project1.bib file to
% see how to organize your references.
\bibliography{project1}
\bibliographystyle{aaai} % sets citation and bib style, do not modify

\end{document}
