
\section{Results}
\label{sec:results}

We hope to have a success rate of a program with specific parameters defined as percentage of a total number of runs for which it reached below error tolerance, and we will further report the average number of generations needed for it to converge. Finally, we hope to report the equation which models the dataset with a mean-squared-error below our tolerance.  

  % replace the second argument below with your filename. I like to
  % place all my figures in a sub-directory to keep things organized
  \includegraphics[width=0.47\textwidth]{figs/results.pdf}

  % *Every* figure should have a descriptive caption.
  \caption{Figure 1}

  % The label is a handle you create so that you can refer to this
  % figure (using the \ref{} command) from other parts of your
  % document. LaTeX automatically renumbers figures and updates
  % references when you recompile, so you should do it this way rather
  % than hard-coding in references. Notice that I've also been
  % creating labels for the various sections in the document; I could
  % use \ref{} command to refer to those sections using their labels
  % too.
  \label{fig:tex}

\end{figure}

\subsection{Creating Tables}
\label{subsec:tables}

Again, refer to \texttt{results.tex} to learn how to create simple
tables (like table~\ref{tab:example}).
\begin{figure}[htb]
  \centering % centers the entire table

  % The following line sets the parameters of the table: we'll have
  % three columns (one per 'c'), each
  % column will be centered (hence the 'c'; 'l' or 'r' will left or
  % right justify the column) and the columns
  % will have lines between them (that's the purpose of the |s between
  % the 'c's).
  \begin{tabular}{|c|c|c|} 
    \hline \hline % draws two horizontal lines at the top of the table
    Column 1 & Column 2 & Column 3 \\ % separate column contents using the &
    \hline % line after the column headers
    $1$ & $3.1$ & $2.7$ \\
    $42$ & $-1$ & $1729$\\
    \hline \hline
  \end{tabular}

  % As with figures, *every* table should have a descriptive caption
  % and a label for ease of reference.
  \caption{An example table.}
  \label{tab:example}

\end{figure}

